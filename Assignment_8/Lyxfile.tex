\batchmode% ===> this file was generated automatically by noweave --- better not edit it
\makeatletter
\def\input@path{{/home/swapnil/Elec_lab/Assignment_8//}}
\makeatother
\documentclass[canadian]{article}
\usepackage[T1]{fontenc}
\usepackage[latin9]{inputenc}
\usepackage[pdftex,a4paper]{geometry}
\geometry{verbose,tmargin=1.5cm,bmargin=0.5cm,lmargin=1.5cm,rmargin=1.5cm,headheight=0.5cm,headsep=0.5cm,footskip=0.5cm}
\usepackage[active]{srcltx}
\usepackage[pdftex]{graphicx}

\makeatletter
%%%%%%%%%%%%%%%%%%%%%%%%%%%%%% Textclass specific LaTeX commands.
\usepackage{noweb}

\makeatother

\usepackage{babel}
\begin{document}

\title{Python Assignment -2}


\author{Swapnil Basak}


\date{EE11B122, IIT Madras}
\maketitle
\begin{abstract}
PUT ABSTRACT HERE
\end{abstract}

\section*{Program}

We include the shebang and import out libraries


\subsection*{Code}

\nwfilename{Lyxfile.nw}\nwbegincode{1}\sublabel{NW2tjBIE-4CALkl-1}\nwmargintag{{\nwtagstyle{}\subpageref{NW2tjBIE-4CALkl-1}}}\moddef{code~{\nwtagstyle{}\subpageref{NW2tjBIE-4CALkl-1}}}\endmoddef\nwstartdeflinemarkup\nwenddeflinemarkup

#! usr/bin/python

from scipy.integrate import quad
from scipy import linspace, cumsum, where
import matplotlib.pyplot as graph
import scipy.special as sp
from numpy import fabs  

\nwnotused{code}\nwendcode{}\nwbegindocs{2}\nwdocspar

\selectlanguage{canadian}%
Now, we set our constants and describe the function that the quad
routine will use. We also declare the vectors that we will be using
throughout this program.


\nwenddocs{}\nwbegincode{3}\sublabel{NW2tjBIE-4eEJrA-1}\nwmargintag{{\nwtagstyle{}\subpageref{NW2tjBIE-4eEJrA-1}}}\moddef{program~{\nwtagstyle{}\subpageref{NW2tjBIE-4eEJrA-1}}}\endmoddef\nwstartdeflinemarkup\nwenddeflinemarkup

# The function that Quad uses
func=lambda x: 1/(1+x**2)
step=0.1
err=1
counter=0
steps = []

samples=sp.arange(0,5+step,step)
results = sp.linspace(0,0,len(samples))
manual= sp.linspace(0,0,len(samples))
estimated_errors, real_errors=[],[]

\nwnotused{program}\nwendcode{}\nwbegindocs{4}\nwdocspar

\selectlanguage{canadian}%
We then decalre the computing function that takes in two arguements
and applie dthe $quad()$ routine and returns the result array.

\nwenddocs{}\nwbegincode{5}\sublabel{NW2tjBIE-Nmg6T-1}\nwmargintag{{\nwtagstyle{}\subpageref{NW2tjBIE-Nmg6T-1}}}\moddef{quadfunc~{\nwtagstyle{}\subpageref{NW2tjBIE-Nmg6T-1}}}\endmoddef\nwstartdeflinemarkup\nwenddeflinemarkup

def integrate(samples,results):
        for i in range(len(samples)):
        # Compute Results with Quad
            results[i]=quad(func,0,samples[i])[0]
        return results
\nwnotused{quadfunc}\nwendcode{}\nwbegindocs{6}\nwdocspar

\selectlanguage{canadian}%
We then declare the function that calculates the above integral using
the trapezoidal step algorithm. It says that

\nwenddocs{}\nwbegincode{7}\sublabel{NW2tjBIE-25UXyi-1}\nwmargintag{{\nwtagstyle{}\subpageref{NW2tjBIE-25UXyi-1}}}\moddef{cumsum~{\nwtagstyle{}\subpageref{NW2tjBIE-25UXyi-1}}}\endmoddef\nwstartdeflinemarkup\nwenddeflinemarkup

def series(new_samples,new_results):
    # Compute manually using recursive summation
    new_results=step*(cumsum(func(new_samples)) - 0.5*(func(min(new_samples))+func(new_samples)))
    return new_results
\nwnotused{cumsum}\nwendcode{}\nwbegindocs{8}\nwdocspar

\selectlanguage{canadian}%
Now we write another function that increases our sample size by halving
the step interval. It returns a tuple both of the new sample set and
result array

\nwenddocs{}\nwbegincode{9}\sublabel{NW2tjBIE-2vwsLi-1}\nwmargintag{{\nwtagstyle{}\subpageref{NW2tjBIE-2vwsLi-1}}}\moddef{ftrap~{\nwtagstyle{}\subpageref{NW2tjBIE-2vwsLi-1}}}\endmoddef\nwstartdeflinemarkup\nwenddeflinemarkup

def ftrap(oldx,oldy):
    # Return double of the sample size
    newx = sp.arange(oldx[0],oldx[-1]+(0.5)*step,0.5*step)
    newy = sp.linspace(0,0,len(newx))
    return (newx,newy)
\nwnotused{ftrap}\nwendcode{}\nwbegindocs{10}\nwdocspar

\selectlanguage{canadian}%
Now we calculate the maximum error for common points in the old and
the new array. We do this using python list slicing in steps and list
comprehension. We also calculate the actual value of the function
by evaluating our sample set with the inbuilt arctan function.

\nwenddocs{}\nwbegincode{11}\sublabel{NW2tjBIE-29DDmX-1}\nwmargintag{{\nwtagstyle{}\subpageref{NW2tjBIE-29DDmX-1}}}\moddef{error~{\nwtagstyle{}\subpageref{NW2tjBIE-29DDmX-1}}}\endmoddef\nwstartdeflinemarkup\nwenddeflinemarkup

def error(old,new):
    # Return the Max absolute value
    if new is None:
        return 1
    new=new[::2]
    err = max([abs(val) for val in new-old])
    return err


results = integrate(samples,results)
preset = sp.arctan(samples)
preset_errors = [fabs(x) for x in results-preset]
\nwnotused{error}\nwendcode{}\nwbegindocs{12}\nwdocspar

\selectlanguage{canadian}%
Before we move on to minimizing error, we store the current values
of the $quad()$ with the preset values of the function. We also plot
a graph to analyze the semilogy plot of the errors between the $quad$
function and the $arctan$ function.

\nwenddocs{}\nwbegincode{13}\sublabel{NW2tjBIE-mnDR7-1}\nwmargintag{{\nwtagstyle{}\subpageref{NW2tjBIE-mnDR7-1}}}\moddef{graphs_1~{\nwtagstyle{}\subpageref{NW2tjBIE-mnDR7-1}}}\endmoddef\nwstartdeflinemarkup\nwenddeflinemarkup

graph.figure(1)
graph.plot(samples,results,'ro',samples,preset,'k')
graph.ylabel('quad fn')

graph.figure(2)
graph.semilogy(samples,preset_errors,'ro')
graph.ylabel('Error')

results = series(samples,manual)
new_samples=samples
new_results=results
\nwnotused{graphs_1}\nwendcode{}\nwbegindocs{14}\nwdocspar

\selectlanguage{canadian}%
We now estimate the error using the method of vectors and recursive
iteration. We state that we want an $\varepsilon$ < $10^{-8}$ or
a maximium of 9 iterations. Each iteration, we call $ftrap()$ and
double the sample set to increase accuracy. We then calculate and
store two different errors - the error between common points and the
error with the preset $arctan()$ function\@.

\nwenddocs{}\nwbegincode{15}\sublabel{NW2tjBIE-4VxdxU-1}\nwmargintag{{\nwtagstyle{}\subpageref{NW2tjBIE-4VxdxU-1}}}\moddef{estimate~{\nwtagstyle{}\subpageref{NW2tjBIE-4VxdxU-1}}}\endmoddef\nwstartdeflinemarkup\nwenddeflinemarkup

# Keep doing until error is below tolerance or counter reaches 9
while(err>pow(10,-7)):
    counter+=1
    if counter>9:
        break
    new_samples,new_results=ftrap(samples,results)
    step = step/2
    steps.append(step)
    new_results = series(new_samples,new_results)
    if counter==1:
        graph.figure(3)
        graph.plot(new_samples,new_results,'ro',samples,preset,'k')
        graph.title('Integration using Trapezoidal Rule') 
    err = error(results,new_results)
    samples=new_samples
    results=new_results
    preset = sp.arctan(new_samples)
    real_errors.append(max([fabs(x) for x in new_results-preset]))
    estimated_errors.append(err)
    print "Running Ftrap iteration", counter 
\nwnotused{estimate}\nwendcode{}\nwbegindocs{16}\nwdocspar

\selectlanguage{canadian}%
Finally, we plot the loglog plot of the two different types of errors
and we see the similarity in their nature.


\nwenddocs{}\nwbegincode{17}\sublabel{NW2tjBIE-31sfAj-1}\nwmargintag{{\nwtagstyle{}\subpageref{NW2tjBIE-31sfAj-1}}}\moddef{graphs_2~{\nwtagstyle{}\subpageref{NW2tjBIE-31sfAj-1}}}\endmoddef\nwstartdeflinemarkup\nwenddeflinemarkup

graph.figure(4)
graph.ylabel(r'Log(Error)')
graph.loglog(steps,estimated_errors,'ro',steps,real_errors,'+')
graph.legend(('Exact error','Estimated error'))
graph.show()
\nwnotused{graphs_2}\nwendcode{}

\nwixlogsorted{c}{{code}{NW2tjBIE-4CALkl-1}{\nwixd{NW2tjBIE-4CALkl-1}}}%
\nwixlogsorted{c}{{cumsum}{NW2tjBIE-25UXyi-1}{\nwixd{NW2tjBIE-25UXyi-1}}}%
\nwixlogsorted{c}{{error}{NW2tjBIE-29DDmX-1}{\nwixd{NW2tjBIE-29DDmX-1}}}%
\nwixlogsorted{c}{{estimate}{NW2tjBIE-4VxdxU-1}{\nwixd{NW2tjBIE-4VxdxU-1}}}%
\nwixlogsorted{c}{{ftrap}{NW2tjBIE-2vwsLi-1}{\nwixd{NW2tjBIE-2vwsLi-1}}}%
\nwixlogsorted{c}{{graphs_1}{NW2tjBIE-mnDR7-1}{\nwixd{NW2tjBIE-mnDR7-1}}}%
\nwixlogsorted{c}{{graphs_2}{NW2tjBIE-31sfAj-1}{\nwixd{NW2tjBIE-31sfAj-1}}}%
\nwixlogsorted{c}{{program}{NW2tjBIE-4eEJrA-1}{\nwixd{NW2tjBIE-4eEJrA-1}}}%
\nwixlogsorted{c}{{quadfunc}{NW2tjBIE-Nmg6T-1}{\nwixd{NW2tjBIE-Nmg6T-1}}}%
\nwbegindocs{18}\nwdocspar

\selectlanguage{canadian}%
\begin{figure}
\includegraphics[scale=0.6]{0_home_swapnil_Elec_lab_Assignment_8_quad.eps}

\caption{Plot of Quad with arctan}
\end{figure}


\begin{figure}
\includegraphics{1_home_swapnil_Elec_lab_Assignment_8_semilog.eps}

\caption{Semilog plot of the error}
\end{figure}


\begin{figure}
\includegraphics[scale=0.6]{2_home_swapnil_Elec_lab_Assignment_8_Trapezoid.eps}

Plot of the trapezoidal series. As we can see, it is remarkably accurate. 

\includegraphics[scale=0.6]{3_home_swapnil_Elec_lab_Assignment_8_error_estimate.eps}

LogLog plots

\caption{Normal plot of the trapezoidal method and LogLog plots of the errors}
\end{figure}

\end{document}
\nwenddocs{}
